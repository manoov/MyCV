\documentclass[a4paper,10pt]{article}
\usepackage[utf8]{inputenc}
\usepackage[hidelinks]{hyperref}
\usepackage{geometry}
\geometry{margin=1in}

\usepackage{enumitem}
\usepackage{xcolor}
\usepackage{tcolorbox}
\usepackage{fontawesome}
\usepackage{array}
\usepackage{bibentry} 
\usepackage{enumitem}

% Section Headers
\newcommand{\sectiontitle}[1]{\vspace{1em}\textbf{\Large #1}\vspace{0.5em}\\\hrule\vspace{1em}}
\newcommand{\cvsection}[1]{%
    \vspace{2mm}
    \begin{tcolorbox}
        \textbf{\large #1}
    \end{tcolorbox}
    \vspace{-4mm}
}

\newcolumntype{L}{>{\raggedright\arraybackslash}X}%
\newcolumntype{R}{>{\raggedleft\arraybackslash}X}%
\newcolumntype{C}{>{\centering\arraybackslash}X}%

% Commands for icon sizing and positioning
\newcommand{\socialicon}[1]{\raisebox{-0.05em}{\resizebox{!}{1em}{#1}}}
\newcommand{\ieeeicon}[1]{\raisebox{-0.3em}{\resizebox{!}{1.3em}{#1}}}

% Font options
\newcommand{\headerfontiii}{\fontfamily{ppl}\selectfont} % Palatino (elegant serif)

\pagestyle{plain}
\setlength{\parindent}{0pt}
\setlength{\parskip}{6pt}

% Header
\begin{document}
\headerfontiii
\begin{center}
    {\Large \textbf{Dr. Manoov R}}\\
    Senior Assistant Professor\\
    Department of Computer Science and Engineering,\\ Vellore Institute of Technology\\
    Vellore, Tamil Nadu, \\India \\
    \href{https://manoov.github.io}{Website: manoov.github.io}
\end{center}

\sectiontitle{Professional Summary}
Dr. Manoov R is a Senior Assistant Professor at the Vellore Institute of Technology (VIT), in Internet of Things (IoT) department of the School of Computer Science and Engineering. He possesses extensive experience in computer science and related technologies, including Machine Learning, Bioinformatics, Data Mining, and Computational Biology. Dr. Manoov is passionate about teaching and integrates practical applications with theoretical concepts to provide students with hands-on experience. He is also skilled in Moodle administration and Git/Docker for development workflows. Beyond academia, Dr. Manoov is committed to social responsibility, having undertaken pro bono consultancy work for NGOs.

\cvsection{Education}
\begin{itemize}[leftmargin=*]
    \item \textbf{Doctor of Philosophy (PhD)}: Thesis Title: Computational Approaches for identifying Micro-RNA Biomarkers in Neoplasms and Drug Resistance.
    \item \textbf{Master of Technology (MTech)}: Thesis Title: “Content based Image Retrieval”
    \item \textbf{Bachelor of Engineering (BE)}: Thesis Title: “Image Enhancement techniques and Filter mechanisms”
\end{itemize}

\cvsection{Professional Experience}
\begin{itemize}[leftmargin=*]
    \item \textbf{Senior Assistant Professor, VIT} (May 2012 - Present)
        \begin{itemize}[leftmargin=*]
            \item Teaching and research in Computer Science and Engineering.
            \item University Moodle LMS Admin team member, focusing on user authentication and course enrollment.
            \item Maintaining 'Teaching with Moodle' resources.
        \end{itemize}
    \item \textbf{Assistant Professor, Velammal Engineering College} (June 2010 - May 2012)
        \begin{itemize}[leftmargin=*]
            \item Delivered lectures and trained in entrepreneurship via the National Entrepreneurship Network.
        \end{itemize}
        \item \textbf{Assistant Professor, SRM University} (Dec 2003 - June 2010)
        \begin{itemize}[leftmargin=*]
        	\item Notable event Organized- A two day “National Workshop on Software Quality Engineering” in association with SPIN, Chennai with 80\% participants from Industry.
        \end{itemize}
        \item \textbf{Assistant Professor, Karunya Institute of Technology and Sciences} (Jan 2003 - Nov 2003)
        \begin{itemize}[leftmargin=*]
        	\item Served as Industry Internship Coordinator of CSE Department.
        \end{itemize}
\end{itemize}

\cvsection{Skills}
\begin{itemize}[leftmargin=*]
    \item Programming: Proficient in R (70\%).
    \item Statistical Analysis: Skilled in Statistics (60\%).
    \item Learning Management Systems: Expertise in Moodle LMS (70\%).
\end{itemize}

\cvsection{Projects}
\begin{itemize}[leftmargin=*]
    \item \textbf{Electronic Data Capture and Clinical Research Data Analytics (2023-24)}: Provided pro bono consultancy services.
\end{itemize}
\cvsection{Publications}
\begin{itemize}[leftmargin=*]
    \item Vishwakarma, S., Premjit, J. P., Balakrishnan, S., \& Manoov, R. (2023, September). Qiskit Simulation of CNOT Equivalent Circuits. In 2023 International Conference on Quantum Technologies, Communications, Computing, Hardware and Embedded Systems Security (iQ-CCHESS) (pp. 1-6). IEEE.
    \item Rajapandy, Manoov., \& Anbarasu, A. (2021). An improved unsupervised learning approach for potential human microRNA–disease association inference using cluster knowledge. Network Modeling Analysis in Health Informatics and Bioinformatics, 10, 1-16.
    \item Jain, H., Yadav, G., \& Manoov, R. (2020). Churn prediction and retention in banking, telecom and IT sectors using machine learning techniques. In Advances in Machine Learning and Computational Intelligence: Proceedings of ICMLCI 2019 (pp. 137-156). Singapore: Springer Singapore.
    \item Chincholkar, S., \& Rajapandy, Manoov. (2020). Fog image classification and visibility detection using CNN. In Intelligent Computing, Information and Control Systems: ICICCS 2019 (pp. 249-257). Springer International Publishing.
    \item Goyal, R., Manoov, R., Sevugan, P., \& Swarnalatha, P. (2020). Securing the data in cloud environment using parallel and multistage security mechanism. In Soft Computing for Problem Solving: SocProS 2018, Volume 2 (pp. 941-949). Springer Singapore.
    \item Sreekant, A., P, S., G, G., \& Rajapandy, Manoov. (2019). Necessity of Machine Learning and Data Visualization Principles in Marketing Investment Management. International Journal of Innovative Technology and Exploring Engineering, 8(6S4).
    \item M, Thanusha., \& R, Manoov. (2018). Secure and Cost Effective User Revocation Mechanism for Data Analysis in the Cloud, 9(4), 442–449. 
    \item Khandelwal, D., \& Manoov, R. (2017, November). Airbag ECU coupled vehicle accident SMS alert system. In 2017 International Conference on Inventive Computing and Informatics (ICICI) (pp. 82-87). IEEE.
\end{itemize}

\cvsection{Faculty Development Programs}
\begin{itemize}[leftmargin=*]
    \item NEP 2020 Orientation Programme, IIT Madras (May 2024)
    \item Invited Guest Talk: "Hands-on Training in the Evaluation process using Moodle LMS"
    \item Organized Faculty Development Program on "Moodle for Python/Java Code Evaluation" (2022).
    \item Invited talk to 100+ newly joined experienced faculties on formative assessment design to enhance student learning during Faculty Orientation Programme (FOP 2022)
    \item AICTE ATAL Academy IoT Training, Mizoram University (February 2021)
    \item AICTE ATAL Academy Blockchain Training, NIT and Bangalore Institute of Technology (2020)
    \item Resource Person on Faculty Development Program on "Effective use of Moodle for Online Classes" with 559 Participants attending online (2020).
    \item Completed certification course on Coursera on "Learning to Teach Online" by UNSW Sydney (The University of New South Wales) (2020).
\end{itemize}

\cvsection{Volunteer and Leadership Roles}
\begin{itemize}[leftmargin=*]
	\item Moodle LMS Admin team member of University since 2017.
	\item Summer Interns Programme: 8-week AIESEC's Global Talent Internship Program for International Students (2023).
	\item Invited talk for faculty members of English department of VIT University at an FDP organised for technology integration for English course activities using Moodle LMS (2022).
	\item Maintaining a webpage titled “Teaching with Moodle” to assist faculty in utilizing Moodle for blended learning at Vellore Institute of Technology. 
    \item Organized a two-day National Workshop on "Software Quality Engineering” at SRM University jointly organised in association with SPIN, Chennai with 80\% participants from Industry (2009).
    \item Led faculty development workshops on Moodle usage and blockchain technologies.
    
\end{itemize}

\end{document}

